\documentclass[]{ctexart}
\usepackage{lmodern}
\usepackage{amssymb,amsmath}
\usepackage{ifxetex,ifluatex}
\usepackage{fixltx2e} % provides \textsubscript
\ifnum 0\ifxetex 1\fi\ifluatex 1\fi=0 % if pdftex
  \usepackage[T1]{fontenc}
  \usepackage[utf8]{inputenc}
\else % if luatex or xelatex
  \ifxetex
    \usepackage{xltxtra,xunicode}
  \else
    \usepackage{fontspec}
  \fi
  \defaultfontfeatures{Mapping=tex-text,Scale=MatchLowercase}
  \newcommand{\euro}{€}
\fi
% use upquote if available, for straight quotes in verbatim environments
\IfFileExists{upquote.sty}{\usepackage{upquote}}{}
% use microtype if available
\IfFileExists{microtype.sty}{%
\usepackage{microtype}
\UseMicrotypeSet[protrusion]{basicmath} % disable protrusion for tt fonts
}{}
\usepackage[left=3cm,right=3cm,top=2cm,bottom=2cm]{geometry}
\ifxetex
  \usepackage[setpagesize=false, % page size defined by xetex
              unicode=false, % unicode breaks when used with xetex
              xetex]{hyperref}
\else
  \usepackage[unicode=true]{hyperref}
\fi
\usepackage[usenames,dvipsnames]{color}
\hypersetup{breaklinks=true,
            bookmarks=true,
            pdfauthor={蔡苗},
            pdftitle={广义加性模型简介},
            colorlinks=true,
            citecolor=blue,
            urlcolor=blue,
            linkcolor=magenta,
            pdfborder={0 0 0}}
\urlstyle{same}  % don't use monospace font for urls
\setlength{\emergencystretch}{3em}  % prevent overfull lines
\providecommand{\tightlist}{%
  \setlength{\itemsep}{0pt}\setlength{\parskip}{0pt}}
\setcounter{secnumdepth}{5}

\title{广义加性模型简介}
\author{蔡苗\footnote{圣路易斯大学公共卫生学院流行病与生物统计系。电子邮件:
  \url{miao.cai@slu.edu}}}
\date{2019-07-01}
\usepackage{setspace}

% Redefines (sub)paragraphs to behave more like sections
\ifx\paragraph\undefined\else
\let\oldparagraph\paragraph
\renewcommand{\paragraph}[1]{\oldparagraph{#1}\mbox{}}
\fi
\ifx\subparagraph\undefined\else
\let\oldsubparagraph\subparagraph
\renewcommand{\subparagraph}[1]{\oldsubparagraph{#1}\mbox{}}
\fi

\begin{document}
\maketitle
\begin{abstract}
广义加性模型是可以假设自变量与因变量或者因变量的函数之间为非线性关系的统计模型。
\end{abstract}

{
\setcounter{tocdepth}{2}
\tableofcontents
}
\clearpage
\doublespacing

\hypertarget{section}{%
\section{引言}\label{section}}

\hypertarget{gam}{%
\subsection{什么是广义加性模型(GAM)}\label{gam}}

\textbf{广义加性模型}(Generalized additive Model,
GAM)是一类不假设预测变量和结果变量之间的关系是线性的模型。GAM最开始由
Hastie and Tibshirani
(\protect\hyperlink{ref-hastie1986generalized}{1986}) 提出。

本稿件主要参考 Larsen (\protect\hyperlink{ref-larsen2015gam}{2015}) 。

\hypertarget{gam-1}{%
\subsection{为什么我们要学会GAM}\label{gam-1}}

\begin{itemize}
\tightlist
\item
  可解读性
\item
  灵活性
\item
  自动选择参数
\item
  正则化
\end{itemize}

\hypertarget{gam-2}{%
\subsection{实现GAM的软件和包}\label{gam-2}}

大部分的统计软件均可以实现GAM,如\href{https://www.sas.com/en_us/home.html}{SAS}和\href{https://www.r-project.org/}{R}。在SAS中可以使用\texttt{PROC\ GAM}或者\texttt{PROC\ TPSPLINE}来实现,在R中可以使用
Wood (\protect\hyperlink{ref-wood2017generalized}{2017})
开发的\texttt{mgcv}包和由 Hastie (\protect\hyperlink{ref-Rgam}{2018})
开发的\texttt{gam}来实现。相较于SAS,R的GAM扩展包的功能和灵活性都更强,因此我们比较推荐使用R的\texttt{mgcv}包来实现GAM。

\hypertarget{section-1}{%
\section{基本概念}\label{section-1}}

\hypertarget{section-2}{%
\subsection{基底函数}\label{section-2}}

基底函数(basis function)。

\hypertarget{section-3}{%
\subsection{有效自由度}\label{section-3}}

有效自由度(Effective Degrees of Freedom, \textbf{EDF})。

\hypertarget{gam-3}{%
\section{GAM的统计理论}\label{gam-3}}

\hypertarget{section-4}{%
\subsection{平滑化}\label{section-4}}

\begin{itemize}
\tightlist
\item
  局部回归(LOESS)
\item
  平滑样条(smoothing splines)
\item
  回归样条
\end{itemize}

\hypertarget{gam-4}{%
\subsection{GAM的估计}\label{gam-4}}

\begin{itemize}
\tightlist
\item
  局部算分算法
\item
  带惩罚项的迭代重新加权最小二乘法(Penalized iterative reweighted least
  squares, \textbf{PIRLS})
\end{itemize}

\textbf{带惩罚项的似然函数}

\[2\log L(\alpha, s_1(x_1), \cdots, s_k(x_k)) - \sum_{k=1}^K\lambda_kU_k\]

其中\(\log L(\alpha, s_1(x_1), \cdots, s_k(x_k))\)为对数似然函数,\(\sum_{k=1}^K\lambda_kU_k\)为惩罚项

\hypertarget{section-5}{%
\subsection{选择平滑参数的方法}\label{section-5}}

\begin{itemize}
\tightlist
\item
  广义交叉验证(Generalized cross validation criterion, \textbf{CGV})
\item
  受限极大似然估计(Restricted Maximum Likelihood, \textbf{REML})
\end{itemize}

\hypertarget{section-6}{%
\subsection{变量选择}\label{section-6}}

\hypertarget{section-7}{%
\subsubsection{单变量选择}\label{section-7}}

\begin{itemize}
\tightlist
\item
  信息值(Information value, IV)
\item
  证据权重(Weight of evidence, WOE)
\end{itemize}

\hypertarget{section-8}{%
\subsubsection{多变量选择}\label{section-8}}

\begin{itemize}
\tightlist
\item
  分步选择法(向前和向后法)
\item
  收缩估计法(shrinkage)
\end{itemize}

\hypertarget{section-9}{%
\section{实际例子}\label{section-9}}

\clearpage

\hypertarget{section-10}{%
\section*{参考文献}\label{section-10}}
\addcontentsline{toc}{section}{参考文献}

\hypertarget{refs}{}
\leavevmode\hypertarget{ref-Rgam}{}%
Hastie, Trevor. 2018. \emph{Gam: Generalized Additive Models}.
\url{https://CRAN.R-project.org/package=gam}.

\leavevmode\hypertarget{ref-hastie1986generalized}{}%
Hastie, T, and R Tibshirani. 1986. ``Generalized Additive Models
Statistical Science.''

\leavevmode\hypertarget{ref-larsen2015gam}{}%
Larsen, Kim. 2015. ``GAM: The Predictive Modeling Silver Bullet.''
\emph{Multithreaded. Stitch Fix} 30.

\leavevmode\hypertarget{ref-wood2017generalized}{}%
Wood, Simon N. 2017. \emph{Generalized Additive Models: An Introduction
with R}. Chapman; Hall/CRC.

\end{document}
